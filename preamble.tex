%      _______   _____ ______   ___  _______   ___                        
%     |\  ___ \ |\   _ \  _   \|\  \|\  ___ \ |\  \                       
%     \ \   __/|\ \  \\\__\ \  \ \  \ \   __/|\ \  \                      
%      \ \  \_|/_\ \  \\|__| \  \ \  \ \  \_|/_\ \  \                     
%       \ \  \_|\ \ \  \    \ \  \ \  \ \  \_|\ \ \  \____                
%        \ \_______\ \__\    \ \__\ \__\ \_______\ \_______\              
%         \|_______|\|__|     \|__|\|__|\|_______|\|_______|              
%                                                                         
%                                                                         
%                                                                         
%  ________  ________  _________  ___  ________  ________   ________      
% |\   __  \|\   __  \|\___   ___\\  \|\   __  \|\   ___  \|\   ____\     
% \ \  \|\  \ \  \|\  \|___ \  \_\ \  \ \  \|\  \ \  \\ \  \ \  \___|_    
%  \ \  \\\  \ \   ____\   \ \  \ \ \  \ \  \\\  \ \  \\ \  \ \_____  \   
%   \ \  \\\  \ \  \___|    \ \  \ \ \  \ \  \\\  \ \  \\ \  \|____|\  \  
%    \ \_______\ \__\        \ \__\ \ \__\ \_______\ \__\\ \__\____\_\  \ 
%     \|_______|\|__|         \|__|  \|__|\|_______|\|__| \|__|\_________\
%                                                             \|_________|
%                                                                         

% --------------------------------------------------------------------
%% CUSTOM PACKAGES
% --------------------------------------------------------------------

% References
\usepackage[square,numbers]{natbib}
%\usepackage{ieeetran}
\usepackage{cleveref}
%\usepackage{cite}

% Font shapes
%\usepackage{cmbright}
\usepackage[T1]{fontenc}
%\usepackage{textcomp}
%\usepackage{anyfontsize}
%\usepackage{lmodern}

% Notation
\usepackage{amsmath}
\usepackage{amssymb}
\usepackage{amsthm}
\usepackage{physics}
\usepackage{siunitx}
\usepackage[official]{eurosym}
\usepackage{xspace}
\usepackage[mathscr]{euscript}
\usepackage{alphabeta}
\usepackage{xfrac}
\usepackage{nomencl}
\usepackage{dutchcal}
\usepackage{MnSymbol}
\usepackage{romanbar}
\makeindex
\makenomenclature

% Tikz
\usepackage{tikz}
\usepackage{pgfplots}
\usetikzlibrary{patterns,decorations.pathmorphing,positioning}
\usetikzlibrary{shapes,arrows}
\usetikzlibrary{positioning}
\usetikzlibrary{intersections, pgfplots.fillbetween}
\usepgfplotslibrary{fillbetween}
\usetikzlibrary{external}
\usepackage{lipsum}
\pgfkeys{/pgf/number format/.cd,1000 sep={\,}}

% Figures
\usepackage{graphicx}
\usepackage{caption}
\usepackage{subcaption}

% Style
\usepackage{framed}
\usepackage{enumitem}
\usepackage{tcolorbox}
\tcbuselibrary{breakable}
%\tcbuselibrary{skins}

% --------------------------------------------------------------------
%% TIKZ / PGF
% --------------------------------------------------------------------

\pgfplotsset{compat=newest} 

% Make sure that TikZ cache files are named properly
\newcommand{\inputtikz}[1]{%
  \tikzsetnextfilename{#1}%
  \input{#1}%
}

% --------------------------------------------------------------------
%% COLORS
% --------------------------------------------------------------------

\definecolor{accent1}{RGB}{61,152,222}
\definecolor{accent2}{RGB}{216,74,59}
\definecolor{accent3}{RGB}{46,204,113}
\definecolor{accent4}{RGB}{241,196,15}
% \definecolor{gray_custom}{}{}

% TUD COLORS
% ----------------------------------

\definecolor{tudCyan}{RGB}{61,152,222}
\definecolor{tudBlack}{RGB}{0,0,0}
\definecolor{tudWhite}{RGB}{255,255,255}

% Basic colors
\definecolor{tudSeaGreen}{RGB}{111,189,165}
\definecolor{tudGreen}{RGB}{39,131,142}
\definecolor{tudDarkBlue}{RGB}{34,70,122}
\definecolor{tudPurple}{RGB}{36,46,131}
\definecolor{tudTurquoise}{RGB}{50,154,179}
\definecolor{tudSkyBlue}{RGB}{130,187,206}

% Accent colors
\definecolor{tudLavender}{RGB}{121,150,180}
\definecolor{tudOrange}{RGB}{216,130,62}
\definecolor{tudWarmPurple}{RGB}{110,50,122}
\definecolor{tudFuchsia}{RGB}{178,72,146}
\definecolor{tudBrightGreen}{RGB}{183,200,34}
\definecolor{tudYellow}{RGB}{247,234,151}

\definecolor{customBlockColor}{RGB}{200,228,250}
\definecolor{customAlertColor}{RGB}{31,112,173}

%\tcbset{skin=enhanced}
\newenvironment{block}[1]{%
    \tcolorbox[colback=accent1!20!,colframe=accent1,%
    title=\textsf{#1}, breakable]}%
    {\endtcolorbox}

\newenvironment{econ}[1]{%
    \tcolorbox[colback=accent3!20!,colframe=accent3,%
    title=\textsf{#1}, breakable]}%
    {\endtcolorbox}

\newenvironment{thmblock}[1]{%
    \tcolorbox[colback=accent2!20!,colframe=accent2,%
    title=\textsf{#1}]}%
    {\endtcolorbox}

% --------------------------------------------------------------------
%% CUSTOM OPTIONS
% --------------------------------------------------------------------

\theoremstyle{definition}
\newtheorem{definition}{Definition}
\newtheorem*{definition_inf}{Definition}

\theoremstyle{plain}
\newtheorem*{question}{Question}
% SiUnitx

\theoremstyle{remark}
\newtheorem*{remark}{Remark}

\DeclareSIUnit\money{\$}
\DeclareSIUnit\year{yr}

% Citations
\bibliographystyle{IEEEtranN}
%\bibliographystyle{IEEEtran}

% --------------------------------------------------------------------
%% MATH MACRO's
% --------------------------------------------------------------------

%\renewcommand{\sfdefault}{cmbr}

% \DeclareSymbolFont{vectors}{OML}{cmm}{b}{it}
\DeclareSymbolFont{tensors}{OML}{cmbrm}{b}{it}
\DeclareSymbolFontAlphabet{\mathtens}{tensors}

\newcommand{\pic}{\ensuremath{\textnormal{\pi}}}   
\newcommand{\ec}{\ensuremath{\mathrm{e}}}          % Euler's constant
\newcommand{\ii}{\ensuremath{\mathrm{i}}}          % Imaginary unit
\newcommand{\jj}{\ensuremath{\mathrm{j}}}          % Split-complex unit
\newcommand{\jjp}{\ensuremath{\mathrm{j}_{_+}}}          % Split-complex unit
\newcommand{\jjm}{\ensuremath{\mathrm{j}_{_-}}}         % Split-complex unit
\newcommand{\cc}{\ensuremath{\mathrm{c}}}           % Speed of light

\newcommand{\lag}{\ensuremath{\mathscr{L}}}
\newcommand{\ham}{\ensuremath{\mathscr{H}}}

\newcommand{\han}{\ensuremath{\zeta}\xspace} % Hyperbolic angle
\newcommand{\hanv}{\ensuremath{\delta}\xspace} % Hyperbolic angular velocity
\newcommand{\amnt}{\ensuremath{A}\xspace} % Amount function
\newcommand{\ininv}{\ensuremath{K}\xspace} % Initial investment; arm
\renewcommand{\real}{\ensuremath{\mathbb{R}}\xspace}

\newcommand{\hyperbolic}{\ensuremath{\mathbf{H}}\xspace}
\newcommand{\quaternions}{\ensuremath{\mathbb{H}}\xspace}
\newcommand{\complex}{\ensuremath{\mathbb{C}}\xspace}
\newcommand{\integer}{\ensuremath{\mathbb{Z}}\xspace}
\newcommand{\sphere}[1]{\ensuremath{\mathbb{S}^{#1}}\xspace}

\newcommand{\lnorm}[1]{\ensuremath{\norm{#1}_L}\xspace}
\newcommand{\ecokin}{\ensuremath{T^*}\xspace}
\newcommand{\ekin}{\ensuremath{T}\xspace}
\newcommand{\epot}{\ensuremath{U}\xspace}
\newcommand{\ecopot}{\ensuremath{U^*}\xspace}
\newcommand{\gpos}{\ensuremath{\vec{q}}\xspace}
\newcommand{\gvel}{\ensuremath{\dot{\vec{q}}}\xspace}
\newcommand{\gmom}{\ensuremath{\vec{p}}\xspace}

\newcommand{\moebiusgroup}{\ensuremath{\text{Möb}}\xspace}
\newcommand{\automorphgroup}[1]{\ensuremath{\mathrm{Aut}(#1)}\xspace}
\newcommand{\pglgroup}[2]{\ensuremath{\mathrm{PGL}(#1, #2)}\xspace}
\newcommand{\pslgroup}[2]{\ensuremath{\mathrm{PSL}(#1, #2)}\xspace}
\newcommand{\glgroup}[2]{\ensuremath{\mathrm{GL}(#1, #2)}\xspace}
\newcommand{\slgroup}[2]{\ensuremath{\mathrm{SL}(#1, #2)}\xspace}
\newcommand{\sogroup}{\ensuremath{\mathrm{SO}(3, \real)}\xspace}
\newcommand{\sugroup}[1]{\ensuremath{\mathrm{SU}(#1)}\xspace} % SU(2) etc
\newcommand{\restlorentzgroup}{\ensuremath{\mathrm{SO^{+}(1, 3, \real)}}}
\newcommand{\spgroup}[2]{\ensuremath{\mathrm{Sp}(#1, #2)}}
\newcommand{\field}{\ensuremath{\mathbb{F}}\xspace}
\newcommand{\firstff}{\ensuremath{\text{\textsf{\textbf{\Romanbar{I}}}}}}
\newcommand{\secondff}{\ensuremath{\firstff\!\firstff}}

\newcommand{\quat}[4]{\ensuremath{#1 + #2\vb{i} + #3\vb{j} + #4\vb{k}}\xspace}
\newcommand{\corresponds}{\ensuremath{\quad \leftrightsquigarrow \quad}}
\newcommand{\conj}[1]{\ensuremath{\bar{#1}}}

\newcommand{\ctran}[1]{\ensuremath{#1^{\dagger}}}
\newcommand{\tran}[1]{\ensuremath{#1^{\top}}}

\newcommand{\inner}[2]{\ensuremath{\langle #1,\, #2\rangle}}

\newcommand{\largefrac}[2]{\frac{\displaystyle #1}{\displaystyle #2}}

\DeclareMathOperator{\sgn}{sgn}

% Vectors, Tensors, Matrices
% Vector notation: bold italic
\renewcommand{\vec}[1]{\vb*{#1}}
\newcommand{\tens}[1]{\ensuremath{\mathtens{#1}}}

\newcommand{\towrite}[1]{{\centering (\texttt{... #1 ...})}}
