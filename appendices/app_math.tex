%
% An Appendix
\chapter{Some math}
A topology is a collection of elements from the power set of a given set. Roughly speaking, a topology gives rise to a notion of `neighborhoods', which further allows for the concepts of convergence and continuity to be properly defined on a set.

Definitions from \cite{Lee2000}

\begin{definition}[Topology]
    A \emph{topology} on a set $X$ is a collection $\mathscr{T}$ of subsets of $X$ called \emph{open sets}, satisfying the following properties:
    \begin{enumerate}[label=\roman*]
        \item $X$ and $\varnothing$ are elements of $\mathscr{T}$.
        \item $\mathscr{T}$ is closed under finite intersections; if $U_1,\ldots,U_n \in \mathscr{T}$, then their intersection $U_1 \cap \ldots \cap U_n$ is in $\mathscr{T}$.
            \item $\mathscr{T}$ is closed under arbitrary unions: if $U_{\alpha \in A}$ is any (finite or infinite) collection of elements of $\mathscr{T}$, then their union $\displaystyle\bigcup_{\alpha \in A}  U_\alpha$ is in $\mathscr{T}$.
    \end{enumerate}
\end{definition}
\begin{definition}[Homeomorphism]
    If $X$ and $Y$ are topological spaces, a \emph{homeomorphism} from $X$ to $Y$ is defined to be a continuous bijective map $\varphi: X\to Y$ with continuous inverse. If there exists a homeomorphism between $X$ and $Y$, they are called homeomorphic or topologically equivalent.
\end{definition}
