% A Real Chapter
\chapter{Hyperbolic geometry and interest}

The core motivation for the application of hyperbolic geometry on the problem of compound interest is the mathematical identity for hyperbolic functions
\begin{equation}
    \ininv e^{\han} = K\cosh(\han) + K\sinh(\han)
    \label{eq:sinhcosh}
\end{equation}
The left part in this equation is equal to the growth of money given a given period interest \han and subject to continuous compounding interest. For a constant interest rate over time $\han = \hanv t$:
\begin{equation}
    \ininv e^{\hanv t} = K\cosh(\hanv t) + K\sinh(\hanv t)
\end{equation}
where $\ininv$ equals the initial investement deposited. Using the hyperbolic functions, the growth of money can be visualised as travelling along a hyperbola in a plane, as visualised in [...], where $\hanv t$ is the so-called \textit{hyperbolic angle}. To define the latter, the notion of a \textit{hyperbolic sector} must be introduced first:

\begin{definition}[Hyperbolic sector]
    A hyperbolic sector is the region of the plane bounded by the rays from the origin to the points $(a, 1/a)$, $(b, 1/b)$, $a, b \in \real$. A hyperbolic sector in standard position has $a = 1$ and $b > 1$. 
    
\end{definition}

\begin{definition}[Hyperbolic angle]
    Consider the rectangular hyperbola $\{(x, 1/x)\ \vert\ x \in \real^+\}$. The hyperbolic angle in standard position is the angle at the origin between the ray $(1, 1)$ and the ray $(x, 1/x)$. The magnitude of this angle is equal to the area of the corresponding hyperbolic sector.
\end{definition}
These definitions are based on the \textit{rectangular hyperbola}, which is defined as the curve $xy = 1$. However, for the usage of hyperbolic functions and the further treatment in this chapter, the \textit{unit hyperbola} will be used as standard, which is described by $x^2 - y^2 = 1$. The only difference between the two is a \ang{45} rotation around the origin in either clockwise or counterclockwise. 

As such, a position vector of any point along the unit hyperbola can be decomposed along the rectangular axes using the hyperbolic functions in the same fashion as one would do for the unit circle with the `regular' trigonometric functions $\cos$ and $\sin$.

\begin{center}
    \texttt{[INSERT HYPERBOLA DIAGRAM]}
\end{center}

If one considers a two-dimensional space where both axes have the units of money (\si{\money}), the accumulation of a unit sum of money as a result of compound interest over time is bound by the unit hyperbola, and the total amount of money outstanding is equal to the sum of the components of the abcissa and ordinate. Likewise, for an arbitrary initial sum \ininv the corresponding hyperbola is described by the equation $x^2 - y^2 = \ininv^2$. The resulting decomposition is given exactly by \cref{eq:sinhcosh}.

This finding giver rise to a first fundamental principle central to this dissertation:

\textit{
The accumulation of money as a result of compound interest is analogous to a hyperbolic rotational motion with constant hyperbolic angular velocity.}

The ultimate purpose of this concept is , as is customary in the field of economic engineering, to establish useful analogies between concepts from economics or finance and (mechanical) engineering. In this case an analogy can be made between finance/actuarial science and rotational mechanics, with the difference that one is concerned with hyperbolic motions as opposed to traditional circular or elliptic motions. They share a great deal of overlap --- this is the driving motivation for the relevance of the analogy --- but also have important discrepancies. These distinctions and theire ramifications on the `hyperbolic rotational mechanics' of finance are discussed in the upcoming sections.

\section{Hyperbolic geometry}
A deeper dive into the formalities of hyperbolic geometry.

\section{General interest accumulation}
The concept of compound interest is extremely old and well-understood and ubiquitous in modern-day finance. However, calculations are usually performed on a discrete basis, with compoundingmonthly, semianually or anuall

\section{Hyperbolic kinematics}
\subsection{Hyperbolic coordinates}

\section{Angular momentum}

\section{Kinetic energy}

\section{Potential energy}
Explore the idea of a potential field capturing the need for people to 
