\chapter{Research proposal}
\label{chap:proposal}
\citet{Mankiw2017} define three roles of money in society:
\begin{enumerate}[itemsep=0.2ex,parsep=0ex,label=(\roman*),topsep=0.2ex]
    \item a \emph{medium of exchange}; i.e. a universal means of exchanging goods and services without the necessity for bartering, 
    \item a \emph{unit of account}; a measuring stick to compare different goods by means of their prices; and
    \item a \emph{store of value}, which allows translating present earnings to the future.
\end{enumerate}
A sound framework in economic engineering should be able to assign  a precise interpretation to each of these roles. The medium of exchange is encapsulated in the `price' component of the price-quantity phase space that underlies the economic engineering theory. Likewise, the role of money as a unit of account is reflected in the symplectic structure of the economic phase space: as shown in \cref{sec:hamilton_eqs}, the symplectic form (which represents a differential amount of revenue `traced out' in the phase plane) provides an isomorphism between the covariant and contravariant tensor components. However, this does not cover other notions to which the unit of account may apply, such as the substitution effect; so this interpretation may have to be refined. Finally, the role as a store of value has been the most problematic in the past because of (i) its inherent temporal aspect and (ii) the idea that earnings (which are analogous to energy) are converted to either a generalized momentum or a generalized position. This `store of value' problem is most apparent when one tries to model investments of any kind; like stocks, bonds, derivatives, debt, and equity, etc.

It has been proven in the past that financial systems (that is, economic systems that include the market dynamics of investments) are an essential component of the larger economic processes, as demonstrated by \citet{Kruimer2021}, \citet{Vos2019} and \citet{VanArdenne2020}. In past economic engineering research, two prevalent interpretations have been assigned to financial instruments: action-angle coordinates (cf. \cref{chap:ee}) and the rotational analogy (cf. \cref{chap:finance_rotation}). However, both problems present some crucial issues:
\begin{itemize}
    \item The \textbf{action-angle approach} is based on a special canonical transformation of the phase space coordinates that are used in physics to facilitate the integration of complex systems or make statements about their periodicity. Their application on financial systems is based on the fact that they consist of dimensionless angle coordinates (`return') and action coordinates with units of money (`investment'). However, this seems to be an ad hoc solution, since the formal applicability of action-angle coordinates is very precise; they can only be used for conservative and conditionally periodic systems \cite{Arnold1989}. For this reason, the angle coordinates have a constant `angular velocity': clearly, this is not at all representative of the return rate in economics. Likewise, the action coordinates are supposed to be functions of first integrals of the system (and therefore time-invariant); again, this does seem unlikely for the investments in dynamic economic systems. 
    
    This approach may have worked in an applied modeling context by producing the correct dynamics, the direct correspondence with investments and return evidently does not hold up to more rigorous standards. This does not mean that there is no place for action-angle coordinates in economic engineering whatsoever, take for example the business cycle analogy that is proposed in \cref{sec:hamilton_eqs}.
    \item The issues with the \textbf{rotational analogy} are more subtle, but they have far-reaching ramifications nevertheless. As \cref{sec:rotational_analogy} already presented, there is definitely something to say for the `rotational' part, i.e. viewing interest as a hyperbolic rotation --- this is the entire point of \cref{chap:finance_rotation}. Basically, one can argue that the kinematics hold, but the application of the dynamics shows some shortcomings: for example, the difference between the arm and the angular momentum, which both have identical units, the interpretation of rotational kinetic energy. Furthermore, duration has been proposed as being equivalent to the mass moment of inertia, but does this does not cover the full character of financial instruments; beside the fact that duration is now considered to be a valuation concept that belongs in frequency-domain analysis \cite{Krabbenborg2021,Kruimer2021}. In the past, the rotational analogy has been used for modeling in a similar fashion as the action-angle coordinates: simply by making the amount invested and the return conjugate variables in the same way as `traditional' prices and quantities are related in economic engineering.
\end{itemize}
Given the importance of financial systems, a solution to this problem will benefit economic engineering research and its applicability to real situations. It can be observed that the quest for suitable analogies to money and financial instruments is hampered by the `unit problem'; that is, many quantities are expressed in terms of money while they play a very different role: for example, the distinction between money as energy (income) and as a quantity or momentum. Whereas dimensional analysis has been an instructive practice in other applications of economic engineering, employing it carelessly in this situation quickly leads to inconsistencies. For example, the rotational analogy requires the continuous transformation between direct earnings on interest to capital; from a fundamental standpoint, these may correspond to quantities of a different nature; other types of financial instruments (e.g. stocks) suggest that this process should be `pulled apart', for it conceals the underlying mechanism of reinvestment.

As such, the aim of this research is to return to find the solution of this problem in the fundamentals of economic engineering as they are presented in \cref{chap:ee}. These fundamentals consist of the concept of utility as energy and the symplectic structure of the economic phase plane. Because of the unit problem, it is often unclear in finance what has the role of products and prices (or even something else); this will naturally appeal to the usage of canonical coordinate transforms of the regular phase space or a Routhian approach where the Hamiltonian and Lagrangian approach are essentially merged. The idea is to find a mapping between these central concepts and the fundamental theory of interest and utility, in particular the work of Irving Fischer \cite{Fisher1906,Fisher1930}.

Other questions that are to be answered pertain to financial equilibria and conservation principles. For example, the classic mechanical equilibrium (Galilean frame of reference) consists of a mass moving at a constant velocity; in economic engineering, this would be a demander with a constant consumer surplus or a producer with a constant producer surplus; consequently, market prices remain constant over time. This and other conservation laws are governed by Noether's theorem (cf. \cref{ssec:noether}). The crucial link between Noether's theorem and the structure behind the Lagrangian and Hamiltonian utility functions will provide insights that would otherwise be shrouded by the unit problem or the potentially complex differential equation that describes the evolution. Usually, the latter is used to understand the nature of the economic system, but the case is made here that analysis of the Lagrangian and Hamiltonian functions is usually more efficient and instructive.

Although the main goal of this research is to improve the understanding of financial systems, the rigorous methods that will be applied may contribute to economic engineering in general as well. For example, the application of manifold theory to economic systems, both from the perspective of the configuration manifold and the cotangent bundle or phase space. Until now, the natural structure of these spaces has not been researched in depth. Additionally, Hamiltonian and Lagrangian analysis are usually not deemed suitable for dissipative systems; but \citet{Hutters2020} demonstrated how Hamiltonian systems can be applied to systems with linear damping. Because real economic systems are always dissipative (just like their mechanical counterparts), this method may be applied to financial systems as well.

Finally, the incorporation of the nature of capital in the economic engineering framework can be used to explain heavily debated concepts such as inflation, monetary policy, and banking regulation. For example, there have been disputes in  macroeconomic theory about whether inflation is endogenous (that is, it arises naturally) or exogenous to economic systems. This is one example where conservation principles prove their value in economic systems.

ma