\chapter{The Lorentz-Minkowski Plane}

\begin{framed}
    \textbf{Chapter structure}
    \begin{enumerate}
        \item Definition
        \item Hyperbolic numbers
        \item Connection with Special relativity
        \item Smith chart
    \end{enumerate}
\end{framed}

It has now been established that for continuous compounding can, from the right perspective, be viewed as a hyperbolic process. Of course, this is the as much the case for compound interest as for other exponential processes, such as radioactive decay, population growth, the spreading of disease etc. However, one might argue that in the case of compound interest, the analogy is particularly useful for a few reasons:
\begin{itemize}
    \item The decomposition into `capital' and `yield' has a direct financial interpretation.
    \item The complete hyperbola consists of two different `sheets': one positive and one negative. These two sheets can be interpreted as the difference between `debit' and `credit'.
\end{itemize}
One sheet (the positive one) of the hyperbola can be seen as a model of the one-dimensional hyperbolic space $\mathbb{H}^1$. More generally, the hyperbolic $n$-space can be seen as a circle with imaginary radius, i.e. $$\mathbb{H}^n = \qty{
x \in \mathbb{R}^{n+1},\ r \in \mathbb{R}^{+}_{*} : \norm{x}^2 = -r}$$. As such, hyperbolic spaces have a constant negative curvature.

\begin{figure}
    \centering
    \begin{tikzpicture}[scale=1.5]
        %\draw[step=0.5, gray, very thin] (-2, -2) grid (2, 2);
        \draw[->] (-2, 0) -- (2, 0) node[right] {$x$};
        \draw[->] (0, -2) -- (0, 2) node[anchor=south] {$y$};
        \draw[domain=0:1.3, variable=\t, thick, red] plot ({cosh(\t)}, {sinh(\t)}) node[anchor=south] {Debit + interest};
        \draw[domain=-1.3:0, variable=\t, thick, red!50] plot ({cosh(\t)}, {sinh(\t)});
        \draw[domain=0:1.3, variable=\t, thick, tudCyan, thick] plot ({-cosh(\t)}, {-sinh(\t)}) node[anchor=north] {Credit + interest};
        \draw[domain=-1.3:0, variable=\t, thick, tudCyan!50] plot ({-cosh(\t)}, {-sinh(\t)});
        \node[tudCyan!50] at (-2, 2) {Credit + discount};
        \node[red!50] at (2, -2) {Debit + discount};
        \node[draw] at (0, 2.7) {$x^2 - y^2 = K^2$};
    \end{tikzpicture}
    \label{fig:credit_debit}
    \caption{Different parts of the complete hyperbola $x^2  - y^2 = K$ each have their distinct financial interpretation.}
\end{figure}