% Second chapter
\chapter{Economic engineering}
Economic engineering is build on analogies between (macro)economic theory and common engineering disciplines such as thermodynamics, circuit theory and (classical) mechanics\footnote{as opposed to more recent theories in relativistic mechanics and quantum mechanics}. Escpecially for the latter, a rich variety of useful analogues can be devised. There are two common classses of interpretations mechanics: \emph{analytical mechanics} including the formulations by Joseph-Louis Lagrange and sir William Rowan Hamilton, and \emph{vectorial mechanics}, better known as \emph{Newtonian mechanics}. The former variant is usually preferred in the field of engineering whereas the analytical mechanics are due to their mathematical elegance powerful interpretative value. Likewise, the theory of economic engineering can be approached in two similar ways. Usually the `Newtonian' approach is given the most attention, but in the case of this work the energy-based approach will prove to be more useful, which is why it will be the starting point of this discussion instead.

\section{Analytical approach}
Analytical mechanics, more specifically \emph{Lagrangian} and \emph{Hamiltonian} mechanics, are built around the definition of special state functions, respectively called the Lagrangian \(\lag\) and the Hamiltonian \(\ham\).

\subsection{Lagrangian mechanics}
Central to the concept of Lagrangian mechanics is the so-called \emph{configuration space} \(M\), which is an \(n\)-dimensional manifold provided with some parameterization called \emph{generalized coordinates} assembled in the vector \(\gpos\). In classical mechanics, these generalized coordinates parameterize completely every allowable position of the system at issue. 

The configuration manifold may just be equal to \(\real^n\), but in more interesting and realistic cases it is often some manifold embedded in \(\real^n\). This is often the result of holonomic constraints, which are constraints that impose a restriction only on the configuration space, but not, for example, on the allowable velocities.

\begin{econ}{Configuration manifold}
    In economic engineering, \(\gpos\) has a similar intuitive notion, namely the stock values of various products, which denote the `position' of some economic system. The name `position' may be misleading when intuitively ported to the familiar three-dimensional space; the configuration space usually has less structure such as the absence of a metric or inner product.
\end{econ}

The Lagrangian \(\lag\) is then defined as a mapping from the \emph{tangent bundle} \(TM\) of the configuration manifold \(M\), optionally paired with a time argument for time-varying problems to the reals,
    \[ \lag : TM \times \real^+ \to \real, \]
    i.e. it takes a generalized position \(\gpos\) and a generalized velocity \(\gvel\) (which live in the tangent space of \(M\)), and a time instance to some scalar value. Since the economic interpretation of \(\gpos\) is a vector of stock levels, its time derivative \(\gvel\) denotes a flow of goods. In classical mechanics, by convention 
    \[\lag = \ecokin - \epot,\] 
    i.e., where \ecokin is the kinetic co-energy\footnote{As will become clear later, the term `co-energy' is used to distinguish between the definition of kinetic energy in terms of the generalized momenta, which is the perspective of Hamiltonian mechanics.} of the system and \epot the potential energy. Generally, the kinetic co-energy is a quadratic form on the tangent space of the configuration manifold, whose coefficients may depend on the generalized position: \(\ecokin = \frac{1}{2}a_{ij}(\gpos)\dot{q}_i\dot{q_j} \). The potential energy is simply a mapping from the configuration manifold to the reals. If \(M\) is a Riemannian manifold and its Lagrangian has the aforementioned form, the system is called \emph{natural} \cite{Arnold1989}.

The crux of the Lagrangian theory is that motions of a mechanical system coincide with \emph{extremals}\footnote{this is a natural extension of the term \emph{extremum} to the calculus of variations.} of the action integral
\[ S = \int^{t_1}_{t_0} \lag(\gpos, \gvel, t) \dd{t}\]
where \(S\) is a quantity named \emph{action} with units of energy-time, or \si{\joule \second}. This is colloquially named the \emph{principle of least action}, or alternatively (and rather confoundingly), \emph{Hamilton's principle}.

\begin{econ}{Lagrangian}
   The lagrangian is analogous to running costs --- the action \(S\) has the units of money in economic engineering. Hence, the extremal of the aforementioned integral coincides with a cost minimization, which fits well into what is considered as rational behaviour of economic agents on whatever scale. 
\end{econ}

Please note that arguably the most fundamental finding in classical mechanics (and also economic engineering) appears here rather subtly, namely that the Lagrangian is a function of only \(\gpos\) and its first derivative \(\gvel\), but no higher derivatives --- that is, the state of the system is completely determined by the generalized positions and generalized velocities. This encodes Newton's famous second law \(F = ma\) already in a very elegant fashion.

The aforementioned principle only is for now only helpful on a conceptual level, because it does not how to arrive at any solutions. For this, a branch of mathematics called the calculus of variations comes to aid, which is concerned with finding extremals of \emph{functionals}\footnote{A functional is a real-valued function on a vector space (of functions.)}, in this case the action \(S\). A necessary condition for \(S\) to attain an extremum is that 
\[ \var{S} = 0,\]
where \(\var{S}\) is called the \emph{first variation} of \(S\). Just like a regular differential can be seen as an infinitesimal perturbation of a function value, the variation is a very small perturbation of a functional by means of a trajectory \(h(t)\). The resulting perturbed functional can then in general be decomposed in a part that varies linearly with \(h\), and a nonlinear part. The requirement for the extremal is that the \emph{linear part vanishes for any} \(h\) \cite{Arnold1989}.

As shown by \citet{Landau1976}, the condition \(\var{S} = 0\) is equivalent to the \emph{Euler-Lagrange} equation
\[ \dv{}{t}\qty(\pdv{\lag}{\gvel}) - \pdv{\lag}{\gpos} = 0. \]
This yields a total of \(n\) second-order differential equations, or equivalently a system of \(2n\) first-order equations. Of course, this also requires \(2n\) initial conditions for a solution to be found.

It was already mentioned that \(\lag = \ecokin - \epot\) in classical mechanics, but what is the economic interpretation behind this equation?
\begin{econ}{Kinetic energy}
    \citet{Mendel2019} sees kinetic (co)-energy as an analogue to surplus --- i.e. an area either above or below the price elasticity curve in a suply-demand diagram.
\end{econ}

\begin{econ}{Potential energy}
    Blabla
\end{econ}

% \begin{block}{Lagrangian system}
%     Let \(M\) be a differentiable manifold, \(TM\) its tangent bundle, and \(\lag: TM \to \real\) a differentiable function. A map \(\gamma: \real \to M\)  is called a motion in the lagrangian system with configuration manifold \(M\) and lagrangian function \(\lag\) if \(\gamma\) is an extremal of the functional
%     \[
%      \Phi(\gamma) =  \int^{t_1}_{t_0} \lag\qty(\dot{\gamma})\dd{t}
% \]
%     where \(\dot{\gamma}\) is the velocity vector \(\dot{\gamma}(t) \in TM_{\gamma(t)}\) \cite{Arnold1989}.
% \end{block}

\subsection{Hamiltonian mechanics}
One of the great advantages of Hamiltonian mechanics is that it admits a much wider range of coordinate transformations. Of course, any selection of the generalized coordinates that parameterizes the admissible motions of the system is equally valid, the generalized velocities are inherently tied to the choice of these coordinates. In Hamiltonian mechanics, the \(p\) and \(q\) coordinates can be chosen completely independent of each other, which is why a larger class of transformations is allowed\footnote{Due to the large variety of transformations, the coordinates for \(q\) and \(p\) may not longer just pertain to a `spatial' component and a `momentum' component, this distincition will then just be a matter of definition.} \cite{Landau1976}.

Canonical transformations are transformations such that Hamilton's equations remain valid in the new coordinate system. As shown by \citet{Landau1976}, each canonical transformation is characterized by a \emph{generating function}. Poisson brackets are invariant with respect to canonical transformations.


\section{Newtonian approach}
