\chapter{Summary and conclusion}
\label{chap:conclusion}

Hyperbolic rotations make their appearance in many mathematical and physical subjects. A prime example is a Lorentz transformation, which is an orthogonal transformation of spacetime in the theory of special relativity. The Lorentz transformations are the isometry group of the (unit) hyperboloid embedded in the Lorentzian space. When embedded in three-dimensional Lorentz space, the hyperboloid exhibits hyperbolic geometry. This is a generalization of Euclidean geometry, where the parallel axiom is modified to relax the uniqueness of parallel lines. Both the Lorentz transformations and non-Euclidean geometry are subsumed in a type of complex transformations called the Möbius transformations. The Möbius transformations form a group under composition called the Möbius group, the automorphism group of the Riemann sphere. Every Möbius transformation can be paired with a Lorentz transformation of four-dimensional spacetime. Furthermore, the Möbius group contains three subgroups, each of which is the isometry group of one of the three types of (non-)Euclidean geometry: hyperbolic, spherical, and Euclidean geometry.

The economic space of prices and stock levels has a natural symplectic structure. An economic system can be seen as a Hamiltonian system, where the Hamiltonian represents the total earnings in the system. Via the Legendre transformation, the economic Hamiltonian is turned into a Lagrangian, which is the difference between the utility gained by trading goods and the utility gained by possessing goods. In the past, two approaches have been proposed in economic engineering to deal with financial instruments: the rotational analogy and action-angle coordinates. The rotational analogy proclaims that financial instruments correspond to rotational mechanical systems because compound interest can be modeled as a hyperbolic rotation. In contrast, action-angle coordinates rely on a particular canonical transformation of the economic phase space.

Proper incorporation of financial systems in economic engineering is of vital importance for its applicability to large-scale economic systems. Both the action-angle approach and the rotational analogy show inconsistencies when subjected to theoretical scrutiny. Instead of these ad hoc methods, a rigorous economic engineering framework must be developed for financial systems, where the fundamentals of the relevant economic theory and the underlying principles of economic engineering, rooted in the theory of classical mechanics, are linked together. Many of the subjects explored in this literature survey, such as symplectic geometry, differential geometry, and group theory are bound to be important tools in the development of said framework, just like they have been for the corresponding theory in physics.

