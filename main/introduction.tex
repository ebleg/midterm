% Introduction
\chapter{Introduction}
\label{chap:intro}
On the eighth page of the October 16, 1929 edition of the New York Times, a small article headed \cite{NYT1929}
\begin{quote}
    \emph{``Fisher Sees Stocks Permanently High; Yale Economist Tells Purchasing Agents Increased Earnings Justify Rise.''} 
\end{quote}
Responsible for this bold claim was Irving Fisher, a very prominent American economist who pioneered  the subject of monetary economics. Unfortunately for him, on Thursday the 29th of October --- merely nine days later --- the Dow Jones Industrial Average dropped by 11 percent, only to lower by another 13 percent on Monday, and then by 12 percent on Tuesday. The result was the Great Depression, an unprecedented financial crisis with dramatic repercussions: in the three years that followed, unemployment rose to 20 percent and industrial production almost halved \cite{gdepression}. 

Of course, this anecdote is not meant to discredit Irving Fischer as an economist, for he has been a major contributor to modern economic theory, but rather to illustrate that even specialists tend to misinterpret the present state of financial markets. This thesis does not aim to solve this issue right away, but to provide a new approach to the interpretation of the financial system; this may help in the understanding of this vital component of modern-day society. This new approach is founded in the recent theoretical framework of economic engineering, developed by prof. em. M. Mendel \cite{Mendel2019}. 

\paragraph{Economic engineering} An introduction to relevant aspects of economic engineering will be provided in \cref{chap:ee}, but the core idea is to include economic systems in the traditional multi-domain modeling framework that (control) engineers use to study systems of widely varying nature (mechanical, electrical, hydraulic, etc.), possibly with the intention of developing a suitable control strategy. Of course, control strategies are vital for economic systems as well --- although this practice would probably be called `policy making' --- either on small scale, such as firms managing their stock levels and revenue in a changing market, or on a macroeconomic scale, e.g. a central bank deciding whether to lower the interest rate or not.

Economic engineering research has been concerned with financial systems before: an economic analogy to the usage of 
\emph{action-angle coordinates} was exploited by \citet{Vos2019} to improve the models used for monetary policy. \citet{Kruimer2021} developed a macroeconomic model of the U.S. economy, where he included bond and equity markets as vital components of the economic machinery by means of the so-called \emph{rotational analogy}. Apart from the fact that both interpretations are manifestly different ways to describe the same concept, they appear to be flawed in some fundamental ways. As such, the necessity arises to reconcile these methods and address their problems in order to find a unifying economic engineering approach to incorporate financial systems; primarily concerning markets for equity and debt and perhaps their related derivatives such as futures and options. 

\paragraph{Current approach} As mentioned, economic engineering currently recognizes two ways to deal with 
`money problems'. Firstly, money is considered analogous to \emph{action}, as (will be explained in \cref{chap:ee}); action is a quantity that represents the integral of energy over time, with units [\si{\joule \second}]. Action-angle coordinates are a choice of coordinates in the phase space that consist of (constant) action coordinates and dimensionless `angles', indicating a \emph{periodic} motion. Secondly, there is the rotational analogy described in \cref{chap:finance_rotation}; which is arguably a bit more flexible than the action-angle coordinates. Again, based on dimensional analysis, an analogy can be made between `money' and angular momentum (in physics, angular momentum and action have the same dimension). Here the role of the angle represents a return or an accumulated interest, and the `arm' of the rotation is the principal of the investment/debt instrument.

\paragraph{Research goal} The core idea is that, instead of \emph{ad hoc} applying the existing theories, to return to the fundamentals of economic engineering and its ties with analytical mechanics to build a more rigorous foundation for future work. Energy and its analogy to utility in economics play a crucial role here, because (i) it lies at the foundation of analytical mechanics as well and (ii) it allows to make the connection with the existing (not strictly financial) theory of economic engineering, just like the multi-domain modeling techniques in engineering are connected through the universal notions of energy and power. Within analytical mechanics, Noether's theorem describes precisely how mathematical symmetries in the general nature of the system expressed in terms of energy (encoded in a special state function called the \emph{Lagrangian}) dictate \emph{conservation laws} that the system must obey. Naturally, perfect conservation of energy and momentum is equally unlikely in both physics and economics, but one strives nevertheless to construct a `platonic ideal', a conserved and isolated financial system to form the basis of the modeling framework.

The goal of this research can therefore be stated as follows: 
\begin{block}{Research goal}
    To develop a new, consistent, and unified framework to interpret debt and equity instruments in economic engineering, using the formal methods of analytical mechanics; in particular Noether's theorem.
\end{block}
\paragraph{Structure of the literature study} That being said, it is clear that the scope of this research is \emph{theoretical}, as its purpose is to expand and refine the current economic engineering framework. Hence, this literature study contains an overview of some related subjects that will hopefully play a role in the development of the theory.

This literature study consists of two parts: 
\begin{itemize}
    \item \textbf{\Cref{part:one}} dives into the very fundamentals of economic engineering and the current approach to financial systems. \Cref{chap:ee} goes in-depth in the fundamentals of analytical mechanics; for all the concepts that are introduced (Lagrangian, kinetic energy, potential energy, etc.) the economic engineering analog is discussed and motivated as well. In this chapter, the action-angle coordinates and Noether's theorem as well. However, it is by no means meant to be an introduction to action-angle coordinates only; for the entire theory of analytical mechanics is bound to play a vital role in the application to financial theory. Secondly, \cref{chap:finance_rotation} argues why the rotational analogy works; i.e. how it can be seen as a hyperbolic rotation. Some closely related concepts such as Lorentzian vector spaces and the algebra of hyperbolic numbers are introduced as well.
    \item \textbf{\Cref{part:two}} contains subjects that are not directly related to economic engineering nor finance, but they have some important connections with the theoretical background of the subjects discussed in \cref{chap:ee,chap:finance_rotation}. \Cref{chap:relativity} provides a quick introduction to the theory of special relativity, for it is the primary physical application and \emph{raison d'être} of the Lorentzian vector spaces. Subsequently, \cref{chap:hyperbolic_geometry} deals with hyperbolic geometry, which is a natural extension of the hyperbolic motions arising from compound interest and the Lorentzian vector spaces. Finally, \cref{chap:moebius_transforms} shows how the group of Möbius transformations (complex fractional linear transformations) subsumes both special relativity, hyperbolic geometry, and even some aspects of analytical mechanics. These connections are most clearly explained using the tools from group theory. Although \cref{chap:relativity,chap:hyperbolic_geometry,chap:moebius_transforms} are not directly required for the discussion in \cref{part:one}, they are presumed to be of great use in future developments.
\end{itemize}



