% Abstract (does not appear in the Table of Contents)
\chapter*{Abstract}%
Financial instruments are currently interpreted in economic engineering as rotational mechanical systems. This is motivated by the fact that compound interest can be viewed as a hyperbolic rotation. Hyperbolic rotations, or squeeze mappings, make various appearances in physics and mathematics; notable examples are the theory of special relativity, Möbius transformations and hyperbolic geometry. All of these subjects are related, because the group of Möbius transformations is isomorphic to the group of restricted Lorentz transformations in special relativity. The Lorentz group is the isometry group of the hyperbolic plane embedded in three-dimensional Lorentz space. 

The hyperbolic rotations find their application in economic engineering because continuous reinvestment of earnings leads to a hyperbolic rotation in the capital-yield plane. Unfortunately, the financial analogy with rotational dynamics shows some severe inconsistencies when applied to real systems. It is therefore proposed to develop a new interpretation of capital and investments. This approach shall be rooted in the fundamentals of economic engineering based on principles from analytical mechanics.




%Two approaches have been used in economic engineering to tackle financial problems up till now: action-angle coordinates and the rotational analogy. However, both methods present some crucial issues. Action-angle coordinates apply exclusively to conservative systems; moreover, their applicability to financial systems is still a matter of debate. Likewise, the rotational analogy has not been applied consistently in past research and some of its aspects remain yet unclear. Hence, the aim of this research is to develop a consistent interpretation of financial systems that is rigorous and applicable to concrete modeling problems. The belief is that this goal can only be achieved by starting from the elementary fundamentals of economic engineering, i.e. the analytical mechanics approach. Therefore, a thorough study of the foundations of the current methods is required to successfully address their shortcomings.
%
%An economic system can be modeled as a Lagrangian system. Hamilton's principle of least action dictates that economic agents attempt to maximize their convenience yield by sacrificing the least amount of economic surplus as possible. This gives rise to the Euler-Lagrange equations that describe the evolution of the economic system. Using the Legendre transform, the Lagrangian economic system can be ported to a Hamiltonian system, which in turn allows for the usage of action-angle coordinates. Noether's theorem links the symmetry properties of the Lagrangian function to the presence of conservation laws. These conservation laws have a concrete economic interpretation that one can exploit to understand complex systems.
%
%The rotational analogy is inspired by the fact that compound interest and exponential returns can be modeled by a hyperbolic rotation in the capital-yield plane. Hyperbolic rotations can be converted to circular rotations through the usage of complex angles. The capital-yield plane belongs to the class of Lorentzian vector spaces. These spaces are inherently connected to special relativity (because the spacetime interval is a Lorentzian distance) and hyperbolic geometry by virtue of the hyperboloid model. Both special relativity, the hyperbolic motions, and hyperbolic geometry are subsumed by the Möbius transformations. Analysis of the group structure of these transformations reveals connections with the aforementioned subjects, but also with the special linear group and the two-dimensional symplectic group.
%
%These closely intertwined mathematical subjects combined with fundamental principles like Noether's theorem will be employed to understand the implications of the choices that have to be made in the pursuit of a consistent interpretation of financial instruments.